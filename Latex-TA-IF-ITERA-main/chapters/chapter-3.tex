\newpage
\chapter{METODE PENELITIAN} \label{Bab III}

\section{Alur Penelitian} \label{III.Alur}
Penelitian ini dilakukan dengan mengikuti tahapan sistematis untuk memastikan hasil analisis determinan \textit{stunting} akurat dan dapat dipertanggungjawabkan. Alur penelitian dimulai dari studi literatur hingga interpretasi hasil menggunakan metode SHAP. Secara garis besar, tahapan penelitian digambarkan dalam diagram alir (\textit{flowchart}) pada Gambar \ref{fig:3.alur}.

\begin{figure}[H]
    \centering
    % Pastikan kamu punya file gambar flowchart di folder figures/
    % Jika belum ada, sementara bisa di-comment dulu baris includegraphics-nya
    \includegraphics[width=0.7\textwidth]{figure/flowchart_penelitian.png}
    \caption{Diagram Alir Penelitian}
    \label{fig:3.alur}
\end{figure}

\section{Penjabaran Langkah Penelitian} \label{III.Jabar Alur}
Berdasarkan Gambar \ref{fig:3.alur}, berikut adalah penjabaran rinci dari setiap tahapan yang dilakukan dalam penelitian ini:

\subsection{Pengumpulan Data} \label{III.Langkah1}
Langkah awal penelitian adalah pengumpulan data sekunder dari Survei Status Gizi Indonesia (SSGI) tahun 2024. Data ini memuat informasi prevalensi \textit{stunting} serta berbagai indikator kesehatan dan lingkungan di tingkat provinsi di Indonesia. Data diperoleh dalam format CSV (\textit{Comma Separated Values}).

\subsection{Pra-pemrosesan Data (\textit{Data Preprocessing})} \label{III.Langkah2}
Data mentah yang diperoleh perlu diproses agar siap digunakan dalam pemodelan. Tahapan ini meliputi:
\begin{enumerate}
    \item \textbf{Pembersihan Data:} Mengonversi tipe data kolom numerik yang terdeteksi sebagai objek (string) menjadi tipe \textit{float}.
    \item \textbf{Penyaringan Data (\textit{Filtering}):} Menghapus baris data yang tidak relevan untuk analisis tingkat provinsi, yaitu data agregat nasional ("Indonesia") dan provinsi pemekaran baru yang datanya belum lengkap (misalnya "Papua Tengah" dan "Papua Pegunungan") untuk menjaga konsistensi model.
    \item \textbf{Pemisahan Variabel:} Memisahkan variabel target ($y$) yaitu \texttt{prevalensi\_stunting\_persen} dan variabel fitur ($X$) yang mencakup indikator determinan lainnya.
\end{enumerate}

\subsection{Pembangunan Model \textit{Random Forest}} \label{III.Langkah3}
Tahap ini melibatkan pembangunan model regresi menggunakan algoritma \textit{Random Forest}. Proses ini mencakup:
\begin{enumerate}
    \item \textbf{Penentuan Parameter (\textit{Hyperparameter Tuning}):} Menggunakan metode \textit{Grid Search} untuk mencari kombinasi parameter terbaik, meliputi jumlah pohon (\textit{n\_estimators}) dan kedalaman maksimum pohon (\textit{max\_depth}).
    \item \textbf{Validasi Silang (\textit{Cross Validation}):} Mengingat jumlah sampel data provinsi yang terbatas, metode \textit{Leave-One-Out Cross Validation} (LOOCV) digunakan untuk memvalidasi performa model secara objektif dan mencegah \textit{overfitting}.
\end{enumerate}

\subsection{Evaluasi Model} \label{III.Langkah4}
Model yang telah dilatih dievaluasi kinerjanya menggunakan metrik statistik untuk mengukur seberapa dekat prediksi model dengan data aktual. Metrik yang digunakan meliputi \textit{Mean Absolute Error} (MAE), \textit{Root Mean Squared Error} (RMSE), dan koefisien determinasi ($R^2$ Score).

\subsection{Interpretasi Model dengan SHAP} \label{III.Langkah5}
Langkah terakhir adalah menginterpretasikan hasil prediksi model \textit{black-box} Random Forest menggunakan \textit{SHapley Additive exPlanations} (SHAP). Analisis ini bertujuan untuk mengetahui kontribusi setiap fitur (determinan) terhadap angka stunting, baik secara global maupun lokal per provinsi.

\section{Alat dan Bahan} \label{III.AlatBahan}

\subsection{Alat} \label{III.Alat}
Penelitian ini dilaksanakan menggunakan perangkat keras dan perangkat lunak dengan spesifikasi sebagai berikut:
\begin{enumerate}[noitemsep]
    \item \textbf{Perangkat Keras:} Laptop dengan prosesor minimal Intel Core i5/AMD Ryzen 5, RAM 8GB, dan penyimpanan SSD untuk komputasi data.
    \item \textbf{Sistem Operasi:} Windows 11 64-bit.
    \item \textbf{Bahasa Pemrograman:} Python versi 3.10 atau lebih baru.
    \item \textbf{Lingkungan Pengembangan (\textit{IDE}):} Visual Studio Code dan Jupyter Notebook.
    \item \textbf{Pustaka (\textit{Libraries}):} 
    \begin{itemize}
        \item \textit{Pandas} dan \textit{NumPy} untuk manipulasi data.
        \item \textit{Scikit-learn} untuk pemodelan \textit{Random Forest} dan evaluasi.
        \item \textit{Matplotlib} dan \textit{Seaborn} untuk visualisasi data.
        \item \textit{SHAP} untuk interpretasi model.
    \end{itemize}
\end{enumerate}

\subsection{Bahan} \label{III.Bahan}
Bahan utama yang digunakan dalam penelitian ini adalah dataset Survei Status Gizi Indonesia (SSGI) Tahun 2024. Dataset ini berisi rekapitulasi indikator gizi dan kesehatan dari seluruh provinsi di Indonesia yang mencakup variabel dependen (prevalensi stunting) dan variabel independen (faktor determinan).

\section{Metode Analisis Data} \label{III.MetodeAnalisis}
Penelitian ini menggunakan pendekatan kuantitatif dengan metode \textit{Machine Learning}. Metode analisis yang digunakan terdiri dari:

\begin{enumerate}
    \item \textbf{Analisis Deskriptif:} Digunakan untuk memahami distribusi data stunting dan korelasi antar variabel melalui visualisasi histogram dan matriks korelasi (\textit{heatmap}).
    \item \textbf{Analisis Prediktif:} Menggunakan algoritma \textit{Random Forest Regressor} untuk memodelkan hubungan kompleks non-linear antara faktor determinan dan prevalensi stunting.
    \item \textbf{Analisis Interpretatif:} Menggunakan nilai SHAP untuk menjelaskan \textit{feature importance} (tingkat kepentingan fitur) dan arah pengaruh (positif/negatif) dari setiap determinan terhadap stunting.
\end{enumerate}

\section{Ilustrasi Perhitungan Evaluasi} \label{III.Ilustrasi}
Untuk mengukur kinerja model regresi, digunakan beberapa rumus evaluasi sebagai berikut:

\begin{enumerate}
    \item \textbf{Mean Absolute Error (MAE):} Rata-rata selisih mutlak antara nilai prediksi dan nilai aktual.
    \begin{equation}
        MAE = \frac{1}{n} \sum_{i=1}^{n} |y_i - \hat{y}_i|
    \end{equation}
    
    \item \textbf{Root Mean Squared Error (RMSE):} Akar kuadrat dari rata-rata kesalahan kuadrat, yang memberikan bobot lebih pada kesalahan besar.
    \begin{equation}
        RMSE = \sqrt{\frac{1}{n} \sum_{i=1}^{n} (y_i - \hat{y}_i)^2}
    \end{equation}
    
    \item \textbf{Koefisien Determinasi ($R^2$):} Mengukur seberapa baik variabilitas data dijelaskan oleh model.
    \begin{equation}
        R^2 = 1 - \frac{\sum (y_i - \hat{y}_i)^2}{\sum (y_i - \bar{y})^2}
    \end{equation}
\end{enumerate}

Dimana:
\begin{itemize}
    \item $n$ = Jumlah sampel data
    \item $y_i$ = Nilai aktual stunting provinsi ke-$i$
    \item $\hat{y}_i$ = Nilai prediksi stunting provinsi ke-$i$
    \item $\bar{y}$ = Rata-rata nilai stunting aktual
\end{itemize}

\section{Rancangan Pengujian} \label{III.Rancang_Uji}
Pengujian model dilakukan menggunakan skema \textit{Leave-One-Out Cross Validation} (LOOCV). Dalam metode ini, dari total $N$ provinsi, model dilatih menggunakan $N-1$ data provinsi dan diuji pada 1 data provinsi yang disisihkan. Proses ini diulang sebanyak $N$ kali sehingga setiap provinsi pernah menjadi data uji tepat satu kali.

Hipotesis hasil yang diharapkan dari pengujian ini adalah:
\begin{enumerate}
    \item Model \textit{Random Forest} mampu memprediksi prevalensi stunting dengan nilai \textit{error} (MAE/RMSE) yang rendah.
    \item Analisis SHAP dapat mengidentifikasi faktor determinan utama (misalnya sanitasi atau pendidikan ibu) yang memiliki dampak paling signifikan terhadap angka stunting di Indonesia tahun 2024.
\end{enumerate}