\newpage
\chapter{METODE PENELITIAN} \label{Bab III}

\section{Alur Penelitian} \label{III.Alur}
Penelitian ini dirancang untuk menganalisis faktor-faktor determinan penyebab \textit{stunting} pada tingkat provinsi di Indonesia dengan memanfaatkan data sekunder \textit{Annual Report} Survei Status Gizi Indonesia (SSGI) tahun 2024. Pendekatan yang digunakan mengombinasikan pemodelan \textit{machine Learning} regresi menggunakan \textit{Random Forest} dan metode interpretasi model SHAP \textit{SHapley Additive exPlanations}. Model ini dibangun menggunakan dataset agregat tingkat provinsi yang memuat berbagai indikator status gizi balita serta faktor determinan yang mencakup aspek kesehatan ibu dan anak, pola asuh, layanan kesehatan, penyakit, perlindungan sosial, dan kondisi sanitasi. Penelitian ini bertujuan untuk mengidentifikasi  faktor determinan utama yang berkontribusi terhadap tingginya prevalensi \textit{stunting} antar provinsi, serta memberikan pemahaman yang bersifat interpretatif sebagai dasar perumusan kebijakan dan intervensi gizi yang lebih tepat sasaran. Secara garis besar, tahapan penelitian digambarkan dalam diagram alir (\textit{flowchart}) pada Gambar \ref{fig:3.alur}.

\begin{figure}[H]
    \centering
    \includegraphics[width=0.8\textwidth]{figure/alur_utama.jpg}
    \caption{Diagram Alir Penelitian}
    \label{fig:3.alur}
\end{figure}

Berdasarkan Gambar \ref{fig:3.alur}, alur penelitian disusun secara sistematis yang diawali dengan tahap identifikasi masalah mengenai urgensi penanganan \textit{stunting} di Indonesia dan kompleksitas faktor determinan yang mempengaruhinya di tingkat provinsi. Tahap ini didukung oleh studi literatur mendalam untuk memahami variabel-variabel determinan multidimensi yang mencakup berbagai aspek, serta metode pengolahan data dan pemodelan \textit{machine learning} yang relevan. Berlandaskan pemahaman tersebut, selanjutnya dilakukan pengumpulan dataset yang bersumber dari Laporan Tahunan Survei Status Gizi Indonesia (SSGI) tahun 2024. Laporan Tahunan SSGI 2024 kemudian digunakan untuk menyiapkan dataset agregat tingkat provinsi yang memuat variabel target (prevalensi \textit{stunting}) dan variabel fitur (faktor determinan). Data mentah yang diperoleh selanjutnya diproses melalui tahapan pra-pemrosesan data yang meliputi pembersihan data, penyaringan data, dan pemisahan variabel target dan fitur. 

Setelah data siap, tahap berikutnya adalah pembangunan model regresi menggunakan algoritma \textit{Random Forest}. Proses ini mencakup penentuan parameter terbaik melalui \textit{Grid Search} dan validasi silang (\textit{Cross Validation}) untuk memastikan performa model yang optimal. Model yang telah dilatih kemudian dievaluasi kinerjanya menggunakan metrik statistik seperti \textit{Mean Absolute Error} (MAE), \textit{Mean Squared Error} (MSE), \textit{Root Mean Squared Error} (RMSE), \textit{Mean Absolute Percentage Error} (MAPE), serta koefisien determinasi ($R^2$ Score). Langkah terakhir adalah menginterpretasikan hasil prediksi model \textit{Random Forest} menggunakan \textit{SHapley Additive exPlanations} (SHAP) untuk mengetahui kontribusi setiap fitur (determinan) terhadap angka \textit{stunting}, baik secara global maupun lokal per provinsi.

\section{Tahapan Penelitian} \label{sec:tahapan}
Sesuai dengan alur yang tertera pada Gambar \ref{fig:3.alur}, penelitian ini dilaksanakan melalui tahapan-tahapan sebagai berikut:

\subsection{Identifikasi Masalah} \label{subsec:identifikasi}
Tahap pertama adalah mengidentifikasi permasalahan utama terkait prevalensi \textit{stunting} di Indonesia yang masih menjadi isu prioritas nasional. Permasalahan difokuskan pada kompleksitas faktor determinan yang bervariasi antarprovinsi dan keterbatasan metode statistik konvensional dalam menangkap hubungan non-linear antar variabel. Oleh karena itu, penelitian ini dirumuskan untuk tidak hanya memprediksi angka stunting, tetapi juga menjelaskan faktor penyebab utamanya menggunakan pendekatan \textit{Machine Learning}.

\subsection{Studi Literatur} \label{subsec:literatur}
Pada tahap ini, dilakukan penelusuran dan kajian terhadap pustaka yang relevan, meliputi:
\begin{enumerate}
    \item \textbf{Teori Determinan Stunting:} Mempelajari kerangka konsep WHO dan UNICEF terkait penyebab malnutrisi (asupan makan, kesehatan, lingkungan, dan pola asuh) untuk menentukan variabel penelitian yang tepat.
    \item \textbf{Metode Machine Learning:} Mengkaji penelitian terdahulu terkait penerapan algoritma \textit{Random Forest} untuk data kesehatan dan metode validasi \textit{Leave-One-Out Cross Validation} (LOOCV) untuk dataset berukuran kecil.
    \item \textbf{Interpretabilitas Model:} Mempelajari metode SHAP (\textit{SHapley Additive exPlanations}) sebagai solusi untuk menjelaskan model \textit{black-box} agar hasil prediksi dapat dipahami secara medis dan kebijakan.
\end{enumerate}

\subsection{Pengumpulan Data} \label{subsec:pengumpulan}
Setelah variabel ditentukan berdasarkan studi literatur, dilakukan pengumpulan data sekunder dari Laporan Publikasi Survei Status Gizi Indonesia (SSGI) tahun 2024. Data diekstraksi dari lampiran laporan resmi menjadi format terstruktur (CSV). Dataset ini mencakup variabel target (prevalensi stunting) dan 56 variabel fitur yang merepresentasikan indikator kesehatan ibu, balita, serta lingkungan dari seluruh provinsi di Indonesia.

\subsection{Pra-pemrosesan Data (\textit{Data Preprocessing})} \label{subsec:preprocessing}
Data mentah yang diperoleh diproses untuk menjamin kualitas input model melalui langkah-langkah:
\begin{enumerate}
    \item \textbf{Pembersihan Data (\textit{Cleaning}):} Mengonversi tipe data numerik dan menangani nilai yang hilang atau tidak konsisten.
    \item \textbf{Seleksi Fitur Berbasis Domain:} Memilih variabel yang relevan berdasarkan kajian literatur dan membuang variabel yang bersifat \textit{confounding} (seperti status gizi lain: \textit{underweight} dan \textit{wasting}) agar model fokus pada determinan penyebab.
    \item \textbf{Penyaringan Wilayah:} Menggunakan 36 provinsi dengan data lengkap dan mengecualikan provinsi pemekaran baru yang datanya belum stabil untuk menjaga validitas pelatihan.
\end{enumerate}

\subsection{Pembangunan Model \textit{Random Forest}} \label{subsec:modeling}
Pada tahap ini, algoritma \textit{Random Forest Regressor} diterapkan untuk membangun model prediksi. Proses ini meliputi:
\begin{enumerate}
    \item \textbf{Optimasi Parameter (\textit{Hyperparameter Tuning}):} Melakukan pencarian parameter optimal (\textit{grid search}) untuk komponen \textit{n\_estimators} (jumlah pohon) dan \textit{max\_depth} (kedalaman pohon) guna mendapatkan akurasi terbaik.
    \item \textbf{Validasi Model (LOOCV):} Menggunakan skema \textit{Leave-One-Out Cross Validation}, dimana model dilatih dan diuji berulang kali sebanyak jumlah provinsi ($N$) untuk memastikan model tidak mengalami \textit{overfitting} akibat sedikitnya jumlah sampel data.
\end{enumerate}

\subsection{Evaluasi Performa Model} \label{subsec:evaluasi}
Kinerja model diukur menggunakan metrik evaluasi regresi untuk melihat tingkat kesalahan prediksi. Metrik yang digunakan adalah \textit{Mean Absolute Error} (MAE) untuk melihat rata-rata kesalahan absolut, \textit{Root Mean Squared Error} (RMSE) untuk melihat sensitivitas terhadap kesalahan besar, dan $R^2$ Score untuk melihat kemampuan model menjelaskan varian data.

\subsection{Interpretasi Hasil dengan SHAP} \label{subsec:shap}
Tahap akhir dan utama adalah penerapan metode SHAP untuk membuka "kotak hitam" model \textit{Random Forest}. Analisis ini menghasilkan:
\begin{enumerate}
    \item \textbf{Interpretasi Global:} Mengidentifikasi fitur mana yang paling dominan mempengaruhi stunting di skala nasional melalui \textit{Summary Plot}.
    \item \textbf{Interpretasi Lokal:} Menganalisis faktor spesifik penyebab tingginya atau rendahnya stunting di provinsi tertentu menggunakan \textit{Waterfall Plot}, sehingga rekomendasi yang dihasilkan lebih tepat sasaran.
\end{enumerate}

\section{Alat dan Bahan} \label{sec:alat_bahan}

\subsection{Alat Penelitian}
Penelitian ini dilaksanakan menggunakan perangkat keras dan perangkat lunak dengan spesifikasi sebagai berikut:
\begin{enumerate}
    \item \textbf{Perangkat Keras:} Laptop dengan prosesor minimal Intel Core i5/AMD Ryzen 5, RAM 8GB, dan penyimpanan SSD.
    \item \textbf{Lingkungan Pengembangan:} Visual Studio Code dan Jupyter Notebook dengan bahasa pemrograman Python 3.10.
    \item \textbf{Pustaka (\textit{Libraries}):} 
    \begin{itemize}
        \item \textit{Pandas} dan \textit{NumPy} untuk manipulasi data.
        \item \textit{Scikit-learn} untuk algoritma \textit{Random Forest} dan evaluasi.
        \item \textit{Matplotlib} dan \textit{Seaborn} untuk visualisasi data.
        \item \textit{SHAP} untuk interpretasi model.
    \end{itemize}
\end{enumerate}

\subsection{Bahan Penelitian}
Bahan utama adalah dataset SSGI Tahun 2024 yang mencakup variabel dependen (prevalensi stunting) dan 56 variabel independen yang dikelompokkan ke dalam pilar determinan sebagai berikut:
\begin{enumerate}
    \item \textbf{Praktik Pengasuhan \& Kesehatan Ibu:} Termasuk risiko kehamilan, kepatuhan TTD, dan akses KB.
    \item \textbf{Pemberian Makan Bayi \& Anak (PMBA):} Termasuk IMD, ASI Eksklusif, dan keragaman diet MPASI.
    \item \textbf{Pelayanan Kesehatan Balita:} Termasuk kelengkapan imunisasi dan pemantauan pertumbuhan.
    \item \textbf{Kesehatan Lingkungan:} Akses air minum, sanitasi, dan pengelolaan limbah layak.
    \item \textbf{Faktor Sosial-Ekonomi:} Termasuk kepemilikan JKN, bantuan sosial, dan pendampingan keluarga.
\end{enumerate}

\section{Ilustrasi Perhitungan Evaluasi} \label{sec:rumus}
Untuk mengukur kinerja model regresi secara kuantitatif, digunakan persamaan sebagai berikut:

\begin{enumerate}
    \item \textbf{Mean Absolute Error (MAE):}
    \begin{equation}
        MAE = \frac{1}{n} \sum_{i=1}^{n} |y_i - \hat{y}_i|
    \end{equation}
    
    \item \textbf{Root Mean Squared Error (RMSE):}
    \begin{equation}
        RMSE = \sqrt{\frac{1}{n} \sum_{i=1}^{n} (y_i - \hat{y}_i)^2}
    \end{equation}
    
    \item \textbf{Koefisien Determinasi ($R^2$):}
    \begin{equation}
        R^2 = 1 - \frac{\sum (y_i - \hat{y}_i)^2}{\sum (y_i - \bar{y})^2}
    \end{equation}
\end{enumerate}
\noindent Dimana $y_i$ adalah nilai aktual, $\hat{y}_i$ adalah nilai prediksi, dan $\bar{y}$ adalah rata-rata nilai aktual.

\section{Rancangan Pengujian} \label{sec:uji}
Pengujian dilakukan untuk membuktikan hipotesis bahwa model \textit{Random Forest} dapat memprediksi prevalensi stunting dengan tingkat kesalahan yang rendah dan analisis SHAP mampu mengisolasi faktor determinan dominan. Validitas pengujian dijamin melalui skema LOOCV dimana setiap provinsi diuji secara independen tanpa pernah dilihat oleh model pada fase pelatihan (\textit{unseen data}).